\section{Electromagnetic Theory}

\subsection{Maxwell's Equations}

\begin{align}
	\nabla \times \vec{E} & = -j\omega\mu\vec{H} \\
	\nabla \times \vec{H} & = j\omega\varepsilon\vec{E} + \vec{J}_0 \\
	\nabla \cdot \vec{D} & = \rho \\
	\nabla \cdot \vec{B} & = 0
\end{align}

\subsection{Green's Functions}

\subsubsection{Electrostatic Green's Function}
Consider the electrostatic Poisson equation
\begin{equation}
	\nabla^2 \phi = -\frac{\rho}{\varepsilon},
\end{equation}
where $\rho$ is the charge density. To find the electric potential due to an arbitrary charge configuration, we can apply the method using the electrostatic Green's function as outlined above. First, we find the impulse response $G(r)$ such that
\begin{equation}
	\nabla^2 G(\vecr) = -\frac{\delta(\vecr)}{\varepsilon}.
\end{equation}
Expanding the Laplace operator in spherical coordinates, we find
\begin{equation*}
	\nabla^2 G = \frac{1}{r^2} \frac{\partial}{\partial r} \left[ r^2 \frac
	{\partial G}{\partial r} \right] = \frac{\partial^2 G}{\partial r^2} + \frac{2}{r} \frac{\partial G}{\partial r} = \frac{\partial^2 (r G)}{\partial r^2}
\end{equation*}
Outside of the point $r = 0$, the Dirac delta function is zero, such that
\begin{equation}
	\frac{\partial^2 (r G)}{\partial r^2} = 0 \qquad r > 0
\end{equation}
Integrating this differential equation twice gives us the general solution $G(r) = a + \frac{b}{r}$. We must have $G \to 0$ as $r \to \infty$, so the coefficient $a = 0$. To find coefficient $b$, we integrate the differential equation over a sphere of radius $R$:
\begin{align*}
	b \iiint_{V} \nabla^2 \left( \frac{1}{r} \right) dV & = - \iiint_V \frac{\delta(r)}{\varepsilon} dV = -\frac{1}{\varepsilon} \\
	\oiint_{\partial V} \nabla \left( \frac{1}{r} \right) \cdot \hat{\vecr} ds & = -\frac{1}{\varepsilon b} \\
	\oiint_{\partial V} \frac{1}{r^2} ds & = \frac{1}{\varepsilon b} \\
	4\pi & = \frac{1}{\varepsilon b}
\end{align*}
Therefore, the electrostatic Green's function becomes
\begin{equation}
	G(\vecr, \vecrp) = \frac{1}{4\pi \varepsilon r}, \qquad r = |\vecr - \vecrp|.
\end{equation}
Then, the electric potential due to an arbitrary charge distribution $\rho(\vecr)$ is given by
\begin{equation}
	\phi(\vecr) = \iiint G(\vecr, \vecrp) \rho(\vecrp) d\vecrp = \frac{1}{4\pi\varepsilon} \iiint \frac{\rho(\vecrp)}{|\vecr - \vecrp|} d\vecrp
\end{equation}

\subsubsection{Electrodynamic Green's Function}
Similarly, we can derive a Green's function which satisfies the scalar Helmholtz equation. It will be shown later that the electrodynamic equations resemble the (vector) Helmholtz equation. Consider
\begin{equation}
	\nabla^2 G(\vecr, \vecrp) + k^2 G(\vecr, \vecrp) = -\delta(\vecr, \vecrp)
\end{equation}
Using the same derivation as before, we can equate the left-hand side to zero for $r > 0$.
\begin{equation}
	\frac{\partial^2 (r G)}{\partial r^2} + k^2 (r G) = 0 \qquad r > 0
\end{equation}
The solution to this differential equation is
\begin{equation*}
	G(r) = \frac{a e^{-j k r}}{r} + \frac{b e^{+j k r}}{r}.
\end{equation*}
By requiring that $G \to 0$ as $r \to \infty$, we find that $b = 0$. To find coefficient $a$, we again integrate over a sphere of radius $R$.
\begin{equation*}
	a \iiint_V \nabla^2 \left( \frac{e^{-jkr}}{r} \right) + k^2 \left( \frac{e^{-jkr}}{r} \right) dV = -1
\end{equation*}
The first part of the integral can be tackled by applying the divergence theorem:
\begin{align*}
	\iiint_V \nabla^2 \left( \frac{e^{-jkr}}{r} \right) dV & = \oiint_{\partial V} \nabla \left( \frac{e^{-jkr}}{r} \right) \cdot \hat{\vecr} ds = 4\pi a^2 \left[ \frac{\partial}{\partial r} \left( \frac{e^{-jkr}}{r} \right) \right]_{r = R} \\
	\lim_{R \to 0} 4\pi a^2 \left[ \frac{\partial}{\partial r} \left( \frac{e^{-jkr}}{r} \right) \right]_{r = R} & = -4\pi
\end{align*}
The second part of the integral is calculated by inspection:
\begin{align*}
	\iiint_V k^2 \left( \frac{e^{-jkr}}{r} \right) dV & = 4\pi k^2 R^2 \int_0^R \frac{e^{-jkr}}{r} dr \\
	\lim_{R \to 0} 4\pi k^2 R^2 \int_0^R \frac{e^{-jkr}}{r} dr & = 0
\end{align*}
Therefore, we find that the coefficient $a = \frac{1}{4\pi}$ and the electrodynamic Green's function is
\begin{equation}
	G(\vecr, \vecrp) = \frac{e^{-jkr}}{4\pi r}, \qquad r = |\vecr - \vecrp|.
\end{equation}

\subsection{Electric Field Integral Equation (EFIE)}
\begin{equation*}
	\nabla \times \nabla \times \vec{E}  = -j\omega\mu (j\omega\varepsilon \vec{E} + \vec{J}) = \nabla (\nabla \cdot \vec{E}) - \nabla^2 \vec{E}
\end{equation*}
Gathering the unknown $\vec{E}$ on the left-hand side, and the source terms involving $\vec{J}$ on the right:
\begin{equation}
	\nabla^2 \vec{E} + k^2 \vec{E} = j\omega\mu\vec{J} - \frac{\nabla (\nabla \cdot \vec{J})}{j\omega\varepsilon}
\end{equation}
This equation has the vector Helmholtz form with wavenumber $k = \nicefrac{\omega}{c} = \omega \sqrt{\mu \varepsilon}$. Using the electrodynamic Green's function derived above, we get an integral equation for the electric field.
\begin{equation}
	\vec{E}(\vecr) = -j\omega\mu \iiint G(\vecr, \vecrp) \left[ 1 + \frac
	{\nabla^\prime \nabla^\prime}{k^2} \right] \vec{J}(\vecrp) d\vecrp
	\label{eq:efie}
\end{equation}
Equation (\ref{eq:efie}) is called the \emph{electric field integral equation} (EFIE) and can also be written more compactly in terms of the operator $\oL$.
\begin{equation}
	\vec{E}(\vecr) = -j\omega\mu (\oL \vec{J})(\vecr),
\end{equation}
where
\begin{equation}
	(\oL\vec{X})(\vecr) = \iiint G(\vecr, \vecrp) \left[ 1 + \frac{\nabla^\prime \nabla^\prime}{k^2} \right] \vec{X}(\vecrp) d\vecrp
\end{equation}